\documentclass[journal]{IEEEtran}
%\documentclass[conference]{IEEEtran}
%\documentclass[a4paper, 11pt]{article}

\usepackage[english]{babel}
\usepackage[T1]{fontenc} 
\usepackage[utf8]{inputenc}
\usepackage{fullpage}
\usepackage{url}
\usepackage{xspace}
\usepackage{eurosym}
\usepackage{lmodern} % for textsc

\usepackage{authblk} % for authors affiliation

\usepackage{hyperref}
\hypersetup{
  colorlinks   = true, %Colours links instead of ugly boxes
  urlcolor     = blue, %Colour for external hyperlinks
  linkcolor    = blue, %Colour of internal links
  citecolor   = blue, %Colour of citations
  breaklinks = true %use \href{url}{\nolinkurl{display URL}} for breakable long URLs
}

% \topmargin = 0pt
% \headheight = 5pt
% \headsep = 10pt
% \textwidth = 410pt
% \textheight = 650pt
% \oddsidemargin = 20pt

\title{A café-philo for computer science:\\motives, experience and next steps}
\date{}

\author[1,3]{Marc Bruyère}
\author[2,3]{Florian Richoux}
\author[4]{Daphné Tuncer}
\author[5]{Marc van der Wal}
\affil[   ]{\it   marc@iij.ad.jp    \quad   florian@richoux.fr   \quad
  daphne.tuncer@enpc.fr \quad marc.vanderwal@afnic.fr}
\affil[1]{IIJ Research Lab, Japan}
\affil[2]{AIST, Japan}
\affil[3]{JFLI, CNRS, Japan}
\affil[4]{École des Ponts ParisTech, France}
\affil[5]{Afnic, France}

\begin{document}
\maketitle

\begin{abstract}
  In  this short  paper,  we report  on our  experience  setting up  a
  monthly  reading group  around  the  socio-technical challenges  and
  ethics of computer science and  digitalization. Our reading group is
  the result  of an  encounter between  four computer  scientists with
  different backgrounds and career paths,  who work in different areas
  of computer systems, and who  live in different places. Beyond these
  factual differences,  we share an  eagerness to talk with  our peers
  not only  about our technical contributions  to digital developments
  but also about their implications on the future of our societies.
\end{abstract}

\section{The need to discuss}

The  incidence of  digital developments  in day-to-day  life has  been
massive over  the last  30 years.  From the  way a  large part  of our
interactions  takes  place  today  to the  way  vital  infrastructures
operate or the way the  world’s geopolitics is shaped, the realization
of digital tools and technologies is far from being a technical matter
only. As practitioners of computer science, both through our education
in college and through our  professional occupation, we, as co-authors
of this paper, acknowledge that  we have directly been contributing to
the construction of this digital era.

Before being computer scientists, we are people who live, work, evolve
in a  set era and  a set system,  and indifferently from  non computer
scientists,   are  recipient   of  the   outcomes  of   these  digital
developments. This  status of  both the  operator of  developments and
their recipient makes us greatly think about what happens with the use
and  the application  of digital  constructs and  technologies in  the
wild. It  is however not  always straightforward to  communicate these
interrogations in our professional communities that often dismiss such
considerations as irrelevant to the purpose of the community.

Although, we do realize that talking with our peers about these issues
is fundamental to broaden a  perspective that can otherwise easily get
defined  by  the  view  of   an  epsilon  percentage  of  the  world’s
population. We  therefore decided  to set  a reading  group up  in the
manner of  a café-philo  for computer  science that  lets participants
share and  express their  opinion on anything  related to  the digital
world. Our discussions range many topics, from the possibilities and 
opportunities of AI for people but also its misuse for purposes that go 
against the common good, to the digital divide, the society of control, and 
the undeniable environmental implications and sustainability issues of widespread 
digitalization.

\section{Reading to debate}

For the  four of  us, the reading  group is a  side project.  It takes
place  in  sessions organized  on  a  monthly  basis. The  setting  is
informal. The objective is to encourage  the free sharing of ideas and
opinions. The time for discussion  is constrained only by the one-hour
slot limit reserved for this activity.

Each  session  is led  by  one  of  the  participants, on  a  rotating
basis. The  lead for the  session presents a  piece of material  to be
discussed. The choice  of the material is entirely up  to the lead. It
can be  a scientific publication,  a book  targeted to a  more general
audience,  an audio  or  a video,  a documentary,  an  article from  a
newspaper, etc., anything that can  trigger a discussion regarding the
role and impact  of digitalization. Sessions are  usually not formally
concluded. The debate remains open  and the topics discussed are aimed
to support each of us with new ideas and perspectives to be applied in
our practice, as well as to consolidate a general reflection.

Our reading  group has been  meeting for  two years during  which many
different topics  were debated. One  of the sessions was  for instance
dedicated to  discussing the workshop  on the environmental  impact of
Internet   applications  and   systems  organized   by  the   Internet
Architecture Board in 2022 \cite{IAB22}.  Other sessions focused on AI
with the recent paper by Michael  Cook on the social responsibility in
the  domain  of Game  AI  \cite{Cook21},  the  book “Weapons  of  Math
Destruction” by  Cathy O’Neil  \cite{WMD16} or  a selection  of papers
presented at the ACM FAccT 2023 conference.

In between sessions, we also  exchange links to interesting content on
a Slack channel.

\section{Implementation challenges}

Over the course of the last two years, we had to make some adjustments
to the way we carry out the  activity, an experience from which we are
also learning.

We reside  in different geographical  areas located in  different time
zones. The sessions take place online which corresponds to morning and
late afternoon calls depending on our  location. The choice of a video
conferencing tool adapted  to our requirements and needs  has not been
straightforward.   We want  something  easily  accessible, secure  and
respectful of private communication.

When we  started the  group, we did  not set any  agenda prior  to the
session.  The discussion  was  entirely unstructured.  We did  however
realize that such a format was  not sustainable in the long run. There
was naturally a need for some directions. We therefore decided to turn
the discussion room into a reading  group by which one person would be
responsible to  present a  piece of  work to the  others from  which a
discussion could be engaged.

It  is essential  for us  to keep  the reading  group informal.  While
highly rewarding  from an  intellectual perspective, this  activity is
(currently)  a  side  project  and  we  believe  that  making  it  too
“academic” might impair  our enthusiasm. At the same  time, it rapidly
turned evident  that having a  goal is  necessary to keep  the dynamic
going. Given that we work in  different domains and sectors, finding a
common target for this initiative can be challenging.

\section{Next steps}

The reading  group has  enabled us  to go  through many  different use
cases, reported  in the press,  in the scientific  literature, through
outreach  initiatives,  etc.  We  aim to  pursue  this  initiative  in
different directions.

We  wish  to  extend  the  sessions   of  the  reading  group  to  new
participants.  In   particular,  we  are  interested   in  kicking-off
collaborations with  practitioners from other disciplines  who have an
interest  in  the  topics  of  the cafe-philo.  In  addition,  we  are
interested in  building on top  an initiative from two  researchers in
mathematics associated with the University  of Cambridge in the UK and
the  University of  Aachen in  Germany  who have  been implementing  a
framework in the form of  a comprehensive manifesto \cite{Chiodo23} to
guide  the   responsible  and   ethical  development   of  data-driven
projects. A  co-author of  the paper  is currently  experimenting with
using the framework in teaching mobility data analytics with Python as
part  of  a final  year  module  at  École  des Ponts  ParisTech.  The
objective  is  to  incentivize  students  to  the  ethical  challenges
associated  with  developing  a  project that  uses  data  to  support
decision-making,   and    make   them   think   out    of   a   purely
performance-oriented  mindset.  In  general,  we  wish  to  develop  a
framework similar to the one proposed by Chiodo and Muller and adapted
to the  specifics of computer science.  In that direction, we  plan to
perform  a  benchmark of  existing  toolsets  for teaching  ethics  to
computer   scientists,   especially   in   French   Doctoral   Schools
(\textit{e.g.,} \cite{MoocEthics}).  We would like to  invite everyone
interested in this initiative to get in touch with us.


\bibliographystyle{alpha} \bibliography{paper}

\begin{IEEEbiographynophoto}{Marc Bruyère}
  began  his  career  in  1996  with  the  Internet  Service  Provider
  Club-Internet.   Over   the  years,  he  has   worked  with  various
  organizations,  including Cisco,  Vivendi  Universal, Credit  Suisse
  First  Boston, Airbus/Dimension  Data, Force10  Networks, and  Dell.
  His academic journey  started in 2012 with a Ph.D.  at the LAAS CNRS
  and  a two-year  PostDoc at  the University  of Tokyo.  His doctoral
  thesis  is  about open-source  OpenFlow  SDN  for Internet  Exchange
  Points (IXPs). Today, he is a senior researcher at IIJ Lab in Japan.

  The rapid  evolution of  internet infrastructure  over the  past few
  decades  has  undeniably transformed  the  landscape  of our  global
  society.   With approximately  65\%  of the  world’s population  now
  connected  online, the  pervasive nature  of this  digital realm  is
  evident.  While this connectivity has brought many opportunities and
  benefits, it has  also given rise to  significant challenges.  Among
  these are the emerging societal  issues, such as the digital divide,
  the society  of control that also  appeared as a unique  response to
  the security  or control  dilemma, and the  undeniable environmental
  implications and sustainability issues.

  Given these challenges,  it becomes essential to  address a research
  question:  How  can we  envision  Internet  and computer  networking
  technologies  to  be  more   attuned  to  human-centric  values  and
  prioritize small, localized scales of operation?

  Computer  networking researchers  can  not ignore  or avoid  ethical
  questions.   Marc has  been  examining and  exploring the  potential
  avenue  in  low-tech  solutions  combined  with  free  software  and
  hardware as  choices.  By advocating  a low-tech approach,  we could
  potentially  reduce  the  environmental  footprint  of  our  digital
  activities.   Using free  software and  hardware ensures  that these
  technologies  remain accessible,  adaptable,  and resilient  against
  monopolistic  tendencies that  can sometimes  stifle innovation  and
  inclusivity.
\end{IEEEbiographynophoto}

\begin{IEEEbiographynophoto}{Florian Richoux}
  is  a senior  research at  the AI  Research Center  of the  National
  Institute of  Advanced Industrial  Science and Technology  in Tokyo,
  Japan.  His research focuses  on Combinatorial Optimization combined
  with Machine Learning, often applied on games.

% The  fruits of  AI research  can unlock  tremendous possibilities  and
% opportunities  for people,  but it  can also  be misused  for purposes
% going against  the common good.  Like the  Laws on Bioethics  voted in
% France in the  late 90’s, we need a legal  frame to delimit boundaries
% between what is  an ethical usage of  AI and what it  is not, defining
% what kind of AI usage we want for our societies.

% This is the purpose of the European AI Act that has been passed by the
% European Parliament on June 2023. This  act proposes a first set of AI
% regulations, but  may not cover all  possible usage of AI  yet: only 6
% months before  its adoption by  the European Parliament,  the European
% Council  adopted a  version of  the text  without taking  into account
% Generative AIs, such  as ChatGPT or DALL·E. As  we know, technological
% improvements in  AI, and in  particular in Machine Learning  (ML), are
% advancing rapidly,  and making difficult for  legislative apparatus to
% keep up.  It is our responsability,  as actors of AI  advancements, to
% keep ourselves  informed of the  new regulations around AI,  to detect
% and analyse  the gaps and  missing rules of  such acts, and  to inform
% both the people and policy makers about what remains to be done.
\end{IEEEbiographynophoto}

\begin{IEEEbiographynophoto}{Daphné Tuncer}
  has been working  in the academic research arena for  15 years.  Her
  research interests are in the  domain of network and computer system
  management.  While she  has mostly  been publishing  pieces of  work
  focusing on technical solutions until now, she has a strong interest
  in the socio-economics of the Internet and digitalization, and their
  impact on  people and the  environment. She reads about,  listens to
  and watches a  lot about ethics, history, and  geopolitics, that are
  all great sources of inspiration  to question research from the what
  for, why  and how perspective.   Daphné is currently with  École des
  Ponts ParisTech,  France. She  initiated the  reading group  in 2021
  with Marc Bruyère.
\end{IEEEbiographynophoto}

\begin{IEEEbiographynophoto}{Marc van der Wal}
  is an  R\&D engineer  at Afnic,  the registry  for domain  names for
  France (.fr) and overseas territories.

  Among his research  and development projects, he  is exploring ideas
  for  fighting   abuse  (spam,  phishing,  etc.)    involving  domain
  names. Some registries, such as  the .nl and .be registries, already
  implement techniques based on machine learning in order to identify,
  at  registration  time, risk  factors  that  may indicate  malicious
  intent.  But  how can  one make  sure that these  tools are  free of
  undesirable side-effects?   His participating in this  reading group
  is one way to find answers to this question.

  He is particularly interested in  explainability of AI models: it is
  an important aspect in the search  for ethics. He is also interested
  in questions regarding privacy and combating digital exclusion.
\end{IEEEbiographynophoto}

\end{document}
